\documentclass[conference]{IEEEtran}
\usepackage{cite}
\usepackage{amsmath,amssymb,amsfonts}
\usepackage{algorithmic}
\usepackage{graphicx}
\usepackage{textcomp}
\def\BibTeX{{\rm B\kern-.05em{\sc i\kern-.025em b}\kern-.08em
    T\kern-.1667em\lower.7ex\hbox{E}\kern-.125emX}}
\begin{document}

\title{Parallel Compilation in the Cloud}

\author{\IEEEauthorblockN{Jamie Willis}
\IEEEauthorblockA{\textit{School of Computer Science} \\
\textit{University of Bristol}\\
jw14896@bristol.ac.uk}
\and
\IEEEauthorblockN{Samuel Russell}
\IEEEauthorblockA{\textit{School of Computer Science} \\
\textit{University of Bristol}\\
sr14978@bristol.ac.uk}
}

\maketitle

\begin{abstract}
We present an application based in the cloud allowing users to compile large C
or C++ projects in parallel, utilising Google App Engine. The application can
compile projects by allowing a user to upload a flattened zip with the source
code and headers. It is available online at \emph{LINK HERE}.
\end{abstract}
\section{Introduction}
Large C or C++ projects often take a very long time to compile, and this is a
large waste of time for developers. Many compilers are not multi-threaded,
since the process of compilation and optimisation must be done in an ordering with
many dependencies. Despite this, however, each file in itself is completely
orthogonal to every other file (until link-time). This results in an ability to
compile multiple files at once by using scripts that launch many compiler
instances in separate threads. This is a good improvement on the compilation
process, but it is limited by the number of processor cores you have.
Additionally, if your compiler \emph{is} multi-threaded, this will place a large
amount of strain on the system.

Our solution is to leverage multiple processors, distributed across the cloud,
each of which can handle the compilation of one or many files depending on how
big the files are. This can prove to be a large advantage for much larger
projects, that might otherwise take over half an hour to compile! This technique
also allows us to leverage multi-threaded compilers whilst also compiling files
in parallel.

% potentially add in more information?

\section{Implementation}
\subsection{Overall Procedure}
When a user uploads a zip file and submits the corresponding compiler and linker
flags to the app, the following process must be undertaken; firstly, the zip
file is unpacked server-side and the GCC macro pre-processor is ran over each of
the source files. This allows us to discard the header files, since they have
already been inserted, at the cost of file size. Secondly, the files are
uploaded onto the ``blob'' storage Google App Engine provides. All of the files
are then placed into the work queue requesting compilation, and, additionally,
a request for linking is placed onto the queue. Thirdly, various micro-service
processors consume work from the queue; they download the relevant source file,
compile it and upload the resulting file or error messages to the blob. At the
same time, the linker process is waiting for all files to be finished and, if
they were all successful, links the application and uploads the executable or
coagulates the error messages and submits them back to the end user.
\subsection{Underlying Architecture}
% What is the language? What are we using, what's actually going on?
\section{Scalability}
\section{Future Work and Limitations}
\subsection{Linux Only Compilation}
\subsection{More Languages}
\subsection{Non-nested Zip Structure}
\section{Conclusion}
%\begin{thebibliography}{00}

%\end{thebibliography}
\end{document}